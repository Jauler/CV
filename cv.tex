%************************************
%
% Cirriculum Vitae - Rytis Karpuška
%
%************************************

%Packages stuff
\documentclass[12]{article}
\input{structure.tex}


\begin{document}

\title{Rytis Karpuška}


\parbox{0.5\textwidth}{ % First block
\begin{tabbing} % Enables tabbing
\hspace{3cm} \= \hspace{4cm} \= \kill % Spacing within the block
{\bf Adresas:} \> Paukščių tako 72,\\ % Address line 1
\> Vilnius, LT12189\\ % Address line 2
{\bf Gimimo data:} \> 1992-09-30\\ % Date of birth 
\end{tabbing}}
\hfill % Horizontal space between the two blocks
\parbox{0.5\textwidth}{ % Second block
\begin{tabbing} % Enables tabbing
\hspace{3cm} \= \hspace{4cm} \= \kill % Spacing within the block
{\bf Mobilusis tel. nr.:} \> +37061476947 \\ % Mobile phone
{\bf El. paštas:} \> \href{mailto:rytis.karpuska@gmail.com}{rytis.karpuska@gmail.com} \\ % Email address
\end{tabbing}}

\section{Apie mane}

Esu Rytis Karpuška, gimiau Vilniuje 1992 metų rugsėjį. Nuo pat vaikystės mane itin domino tikslieji mokslai.
Dar būdamas 11-12 metų, pradėjau domėtis fizika, bet ilgainiui iš visų tuo metu žinotų fizikos sričių labiausiai susidomėjau elektronika.
Pradėjau lituoti elementarias schemas, garso stiprintuvus.
Būdamas 13-14 metų aptikau mikrovaldiklius, ir, pagal pavyzdį, pasirašiau pirmą savo programą.
Tai buvo 7-10 eilučių asemblerio kalba parašytas kodas, kuris uždegdavo LED šviestuką.
Tai mane labai stipriai sudomino, ir nuo to laiko praktiškai nuolat savo malonumui vykdau vienokį ar kitokį projektą, susijusį su programavimu, inžinerija.


\section{Išsilavinimas}

\tabbedblock{
	\bf{2011 - 2015} \> VU Bakalauras, Programų sistemos - \href{http://www.vu.lt}{Vilniaus Universitetas} \\[5pt]
	\>Pirmas bei antras semestrai - 90+\% Vidurkis\\
	\>Trys semestrai su stipendija\\
	\>\+
}

\tabbedblock{
	\bf{2011} \> Andrew Ng vedamas internetinis "Machine Learning" kursas. \textit{http://ml-class.org}\\
	\>\+
}

\section{Profesinė patirtis}

\job
{2011 -}{Dabar}
{UAB "Elektromotus", Žirmūnų g. 68, Vilnius, Lietuva}
{http://www.elektromotus.eu}
{Architektas / Programuotojas}
{
UAB "Elektromotus" įsidarbinau 2011 Liepą, kaip C programuotojas.
Nuo to laiko teko dirbti prie įvairių projektų.}


\section{Projektai}

\job
{2007}{}
{Kliūčių išvengiantis robotas "Infobalt 2007" parodai.}
{http://blog.elektronika.lt/robotai/2007/10/28/reportazai-is-infobalt2007/}
{http://blog.elektronika.lt/robotai/2007/10/28/reportazai-is-infobalt2007/}
{Tai yra įdomiausias darbas darytas dar vaikystėje/paauglystėje.
Robotukas buvo valdomas AVR ATmega valdiklio, turėjo keletą sensorių ir varikliukų.\\
\rule{0mm}{5mm}\textbf{Technologijos:} AVR, GCC.}


\job
{2011 -}{2012}
{Akumuliatorių valdymo sistema "Emus BMS"}
{http://www.elektromotus.lt/lt/produktai/bms.html}
{http://www.elektromotus.lt/lt/produktai/bms.html}
{Prie šio projekto dirbau kaip UAB "Elektromotus" darbuotojas. Tai yra pagrindinis įmonės produktas, dar iki šiol vystomas ir parduodamas.
"Emus BMS" yra sistema, skirta kontroliuoti bei apsaugoti įvairius ličio ir nikelio pagrindu pagamintus akumuliatorius nuo perkaitimų, perkrovų, per didelių iškrovų ir pan.
Ši sistema naudojama vidutiniuose ir dideliuose akumuliatorių baterijose.
Sėkmingiausias panaudojimas - elektrinis bolidas "ACCIONA" dakaro ralyje.\\
\rule{0mm}{5mm}\textbf{Technologijos:} AVR, svn, GCC, Linux, Qt4.}

\job
{2012 - }{2013}
{Automobilio "Smart fourtwo" perdarymas elektriniu.}
{http://grynas.delfi.lt/tv/lietuviu-perdarytas-elektromobilis-100-km-nuvaziuoja-uz-7-litus.d?id=61776195}
{http://grynas.delfi.lt/tv/lietuviu-perdarytas-elektromobilis-100-km-nuvaziuoja-uz-7-litus.d?id=61776195}
{Tai dar vienas "Elektromotus" projektas, prie kurio dirbau kaip pagrindinis programuotojas.
Šio projekto metu suprojektavau ir suprogramavau automobilio "smart" ECU sistemą, kuri valdė
variklį, pavarų dėžę, aušinimą, komunikavo su likusiomis sistemomis.\\
\rule{0mm}{5mm}\textbf{Technologijos:} ARM-Cortex M3, Linux, GCC, git.}

\job
{2014 -}{2015}
{Sensorių tinklas geležinkelių infrastruktūrai stebėti.}
{}
{}
{
Tai projektas, prie kurio dirbau ilgiausiai būdamas UAB "Elektromotus" darbuotoju.
Šio projekto metu, kartu su komanda, suprojektavome ir suprogramavome sistemą, skirtą geležinkelio aukštos įtampos kabelio vibracijoms stebėti.
Dėl aukštų reikalavimų elektros suvartojimui buvo sukurtas reguliuojamos latencijos, žvaigždės topologijos, 2.4Ghz radio tinklo protokolas, kuris buvo įgyvendintas ir ARM-Cortex M3 valdiklio aplinkoje, ir kaip linux branduolio tvarkyklė.
Sistema taip pat suteikia galimybę nuotoliniu būdu konfigūruoti, parsisiųsti matavimus bei stebėti jos darbą.\\
\rule{0mm}{5mm}\textbf{Technologijos:} ARM-Cortex M3, Linux kernel device drivers, GCC, git, MEMS, 2.4Ghz radio, Raspberry pi, GSM, VPN, VPS.}

\job
{2013 -}{Dabar}
{Ketursraigtis (ang.: \textit{Quadcopter}).}
{https://github.com/jauler/Quadcopter}
{https://github.com/jauler/Quadcopter}
{Tai yra laisvalaikio projektas.
Ketursraigtyje naudojamas kvaternionais paremtas kampinės pozicijos skaičiavimas bei nuosekliai sujungti PID valdikliai valdymui.
Valdymo komandos siunčiamos per mobiliuosius 3G, 4G tinklus.
Šiuo metu atliekami pirmieji skrydžio bandymai.\\
\rule{0mm}{5mm}\textbf{Technologijos:} ARM-Cortex M4, GCC, git, MEMS, Raspberry pi, GSM.}


\section{Programavimo įgūdžiai}

\skillgroup{Programavimo kalbos}
{
	\textit{Assembler, C, C++} - 4 metų patirtis profesinėje srityje, 8 įskaitant ir laisvalaikio veiklą.\\
	\textit{Octave, Matlab, Python, Bash} - Dažnas pagalbinių ar pagrindinių script'ų rašymas profesinėje ir asmeninėje veikloje.\\
	\textit{Java, Ruby, html, css, sql, latex} - Naudota asmeniniais tikslais. \\

}

\skillgroup{Kiti įgūdžiai}
{
	\textit{Linux} - 7 metai asmeninio naudojimo, taip pat linux tvarkyklių rašymas profesiniais ir asmeniniais tikslais.\\
	\textit{git, svn} - 4 metus naudota profesiniais bei asmeniniais tikslais.\\
	\textit{vim} - 2 metus asmeniniais ir profesiniais tikslais.\\
}


\section{Interesai}

\interestsgroup{
	\interest{Įterptinės sistemos, 3D grafika, mašinų mokymasis;}
	\interest{"Encounter" urbanistiniai žaidimai;}
	\interest{Vairavimas;}
}


\end{document}



