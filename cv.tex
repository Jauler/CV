%************************************
%
% Cirriculum Vitae - Rytis Karpuška
%
%************************************

%Packages stuff
\documentclass[12]{article}
%%%%%%%%%%%%%%%%%%%%%%%%%%%%%%%%%%%%%%%%%
% Wilson Resume/CV
% Structure Specification File
% Version 1.0 (22/1/2015)
%
% This file has been downloaded from:
% http://www.LaTeXTemplates.com
%
% License:
% CC BY-NC-SA 3.0 (http://creativecommons.org/licenses/by-nc-sa/3.0/)
%
%%%%%%%%%%%%%%%%%%%%%%%%%%%%%%%%%%%%%%%%%

%----------------------------------------------------------------------------------------
%	PACKAGES AND OTHER DOCUMENT CONFIGURATIONS
%----------------------------------------------------------------------------------------

\usepackage[a4paper, hmargin=25mm, vmargin=30mm, top=20mm]{geometry} % Use A4 paper and set margins

\usepackage{fancyhdr} % Customize the header and footer

\usepackage{lastpage} % Required for calculating the number of pages in the document

\usepackage{hyperref} % Colors for links, text and headings

\setcounter{secnumdepth}{0} % Suppress section numbering

%\usepackage[proportional,scaled=1.064]{erewhon} % Use the Erewhon font
%\usepackage[erewhon,vvarbb,bigdelims]{newtxmath} % Use the Erewhon font
\usepackage[utf8]{inputenc} % Required for inputting international characters
\usepackage[T1]{fontenc} % Output font encoding for international characters

\usepackage{fontspec} % Required for specification of custom fonts
\setmainfont[Path = ./fonts/,
Extension = .otf,
BoldFont = Erewhon-Bold,
ItalicFont = Erewhon-Italic,
BoldItalicFont = Erewhon-BoldItalic,
SmallCapsFeatures = {Letters = SmallCaps}
]{Erewhon-Regular}

\usepackage{color} % Required for custom colors
\definecolor{slateblue}{rgb}{0.17,0.22,0.34}

\usepackage{sectsty} % Allows customization of titles
\sectionfont{\color{slateblue}} % Color section titles

\fancypagestyle{plain}{\fancyhf{}\cfoot{\thepage\ of \pageref{LastPage}}} % Define a custom page style
\pagestyle{plain} % Use the custom page style through the document
\renewcommand{\headrulewidth}{0pt} % Disable the default header rule
\renewcommand{\footrulewidth}{0pt} % Disable the default footer rule

\setlength\parindent{0pt} % Stop paragraph indentation

% Non-indenting itemize
\newenvironment{itemize-noindent}
{\setlength{\leftmargini}{0em}\begin{itemize}}
{\end{itemize}}

% Text width for tabbing environments
\newlength{\smallertextwidth}
\setlength{\smallertextwidth}{\textwidth}
\addtolength{\smallertextwidth}{-2cm}

\newcommand{\sqbullet}{~\vrule height 1ex width .8ex depth -.2ex} % Custom square bullet point definition

%----------------------------------------------------------------------------------------
%	MAIN HEADER COMMAND
%----------------------------------------------------------------------------------------

\renewcommand{\title}[1]{
{\huge{\color{slateblue}\textbf{#1}}}\\ % Header section name and color
\rule{\textwidth}{0.5mm}\\ % Rule under the header
}

%----------------------------------------------------------------------------------------
%	JOB COMMAND
%----------------------------------------------------------------------------------------

\newcommand{\job}[6]{
\begin{tabbing}
\hspace{2cm} \= \kill
\textbf{#1} \> \href{#4}{#3} \\
\textbf{#2} \>\+ \textit{#5} \\
%\begin{minipage}{\smallertextwidth}
%\vspace{2mm}
#6
%\end{minipage}
\end{tabbing}
%\vspace{2mm}
}


%----------------------------------------------------------------------------------------
%	HOBBY PROJECT COMMAND
%----------------------------------------------------------------------------------------

\newcommand{\hobbyproject}[2]{
\begin{tabbing}
\hspace{2cm} \= \kill
\textbf{#1} \> #2 \\
\end{tabbing}
}


%----------------------------------------------------------------------------------------
%	SKILL GROUP COMMAND
%----------------------------------------------------------------------------------------

\newcommand{\skillgroup}[2]{
\begin{tabbing}
\hspace{5mm} \= \kill
\sqbullet \>\+ \textbf{#1} \\
\begin{minipage}{\smallertextwidth}
\vspace{2mm}
#2
\end{minipage}
\end{tabbing}
}

%----------------------------------------------------------------------------------------
%	INTERESTS GROUP COMMAND
%-----------------------------------------------------------------------------------------

\newcommand{\interestsgroup}[1]{
\begin{tabbing}
\hspace{5mm} \= \kill
#1
\end{tabbing}
\vspace{-10mm}
}

\newcommand{\interest}[1]{\sqbullet \> \textbf{#1}\\[3pt]} % Define a custom command for individual interests

%----------------------------------------------------------------------------------------
%	TABBED BLOCK COMMAND
%----------------------------------------------------------------------------------------

\newcommand{\tabbedblock}[1]{
\begin{tabbing}
\hspace{2cm} \= \hspace{4cm} \= \kill
#1
\end{tabbing}
}



\begin{document}

\title{Rytis Karpuška}


\parbox{0.5\textwidth}{ % First block
\begin{tabbing} % Enables tabbing
\hspace{3cm} \= \hspace{4cm} \= \kill % Spacing within the block
{\bf Adresas:} \> Paukščių tako 72,\\ % Address line 1
\> Vilnius, LT12189\\ % Address line 2
{\bf Gimimo data:} \> 1992-09-30\\ % Date of birth 
\end{tabbing}}
\hfill % Horizontal space between the two blocks
\parbox{0.5\textwidth}{ % Second block
\begin{tabbing} % Enables tabbing
\hspace{3cm} \= \hspace{4cm} \= \kill % Spacing within the block
{\bf Mobilusis tel. nr.:} \> +37061476947 \\ % Mobile phone
{\bf El. paštas:} \> \href{mailto:rytis.karpuska@gmail.com}{rytis.karpuska@gmail.com} \\ % Email address
\end{tabbing}}

\section{Apie mane}

Esu Rytis Karpuška, gimiau Vilniuje 1992 metų rugsėjį. Nuo pat vaikystės, mane labai domina techniniai dalykai.
Dar būdamas 11-12 metų, pradėjau domėtis fizika, bet ilgainiui iš visų tuo metu žinotų fizikos sričių labiausiai susidomėjau elektronika.
Pradėjau lituoti elementariausias schemas, garso stiprintuvus.
Būdamas 13-14 metų aptikau mikrovaldiklius, ir, pagal pavyzdį, pasirašiau pirmą savo programą.
Tai buvo 7-10 eilučių asemblerio kalba parašytas kodas, kuris sugebėjo uždegti LED šviestuką.
Tai mane labai stipriai sudomino, ir nuo to laiko praktiškai visada turėjau vienokį ar kitokį hoby projektą, kuris susijęs su programavimu, inžinerija ar pan.


\section{Išsilavinimas}

\tabbedblock{
	\bf{2011 - 2015} \> VU Bakalauras, Programų sistemos - \href{http://www.vu.lt}{Vilniaus Universitetas} \\[5pt]
	\>Pirmas bei antras semestrai - 90+\% Vidurkis\\
	\>Trys semestrai su stipendija\\
	\>\+
}

\tabbedblock{
	\bf{2011} \> Andrew ng vedamas internetinis "Machine Learning" kursas. \textit{http://ml-class.org}\\
	\>\+
}

\section{Profesinė patirtis}

\job
{2011 -}{Dabar}
{UAB "Elektromotus", Žirmūnų g. 68, Vilnius, Lietuva}
{http://www.elektromotus.eu}
{Architektas / Programuotojas}
{
UAB "Elektromotus" įsidarbinau 2011 Liepą, pažįstamo elektroniko kvietimu, kaip C programuotojas.
Nuo to laiko teko dirbti prie įvairių projektų, šiek tiek vadovauti komandai (nors ir gana fiktyviai).
}


\section{Projektai}

\job
{2007}{}
{Kliučių išvengiantis robotas "Infobalt 2007" parodai.}
{http://blog.elektronika.lt/robotai/2007/10/28/reportazai-is-infobalt2007/}
{http://blog.elektronika.lt/robotai/2007/10/28/reportazai-is-infobalt2007/}
{Tai yra ryškiausias darbas darytas dar vaikystėje/paauglystėje.
Robotukas buvo valdomas AVR ATmega valdiklio turėjo keletą sensorių ir varikliukų.\\
\rule{0mm}{5mm}\textbf{Technologijos:} AVR, GCC.}


\job
{2011 -}{2012}
{Akumuliatorių valdymo sistema "Emus BMS"}
{http://www.elektromotus.lt/lt/produktai/bms.html}
{http://www.elektromotus.lt/lt/produktai/bms.html}
{Prie šio projekto dirbau kaip UAB "Elektromotus" darbuotojas. Tai yra pagrindinis įmonės produktas, dar iki šiol vystomas ir parduodamas.
"Emus BMS" yra sistema skirta kontroliuoti bei apsaugoti įvairius ličio ir nikelio pagrindu pagamintus akumuliatorius nuo perkaitimų, perkrovų, perdidelių iškrovų ir pan.
Ši sistema naudojama vidutiniuose ir dideliuose akumuliatorių baterijose.
Ryškiausias panaudojimas - elektrinis bolidas "ACCIONA" dakaro ralyje.\\
\rule{0mm}{5mm}\textbf{Technologijos:} AVR, svn, GCC, Linux, Qt4.}

\job
{2012 - }{2013}
{Automobilio "Smart fourtwo" perdarymas elektriniu.}
{http://grynas.delfi.lt/tv/lietuviu-perdarytas-elektromobilis-100-km-nuvaziuoja-uz-7-litus.d?id=61776195}
{http://grynas.delfi.lt/tv/lietuviu-perdarytas-elektromobilis-100-km-nuvaziuoja-uz-7-litus.d?id=61776195}
{Tai dar vienas "Elektromotus" projektas, prie kurio dirbau, kaip pagrindinis programuotojas.
Šio projekto metu suprojektavau ir suprogramavau automobilio "smart" ECU sistemą, kuri valdė
variklį, pavarų dėžę, aušinimą, komunikavo su likusiomis sistemomis.\\
\rule{0mm}{5mm}\textbf{Technologijos:} ARM-Cortex M3, Linux, GCC, git.}

\job
{2014 -}{2015}
{Sensorių tinklas geležinkelių infrastruktūrai stebėti.}
{}
{}
{
Tai projektas, prie kurio dirbau ilgiausiai kaip UAB "Elektromotus" darbuotojas.
Šio projekto metu, kartu su komanda, suprojektavome ir suprogramavome sistemą skirtą geležinkelio aukštos įtampos kabelio vibracijoms stebėti.
Dėl aukštų reikalavimų elektros suvartojimui buvo sukurtas reguliuojamos latencijos, žvaigždės topologijos, 2.4Ghz radio tinklo protokolas, kuris buvo įgyvendintas ir ARM-Cortex M3 valdiklio aplinkoje, ir kaip linux branduolio tvarkyklė.
Sistema taip pat suteikia galimybę nuotolinių būdu konfigūruoti, parsisiūsti matavimus, bei stebėti jos darbą.\\
\rule{0mm}{5mm}\textbf{Technologijos:} ARM-Cortex M3, Linux kernel device drivers, GCC, git, MEMS, 2.4Ghz radio, Raspberry pi, GSM, VPN, VPS.}

\job
{2013 -}{Dabar}
{Ketursraigtis (ang.: \textit{Quadcopter}).}
{https://github.com/jauler/Quadcopter}
{https://github.com/jauler/Quadcopter}
{Tai yra hoby projektas, kurį darau laisvalaikiu.
Ketursraigtyje naudojamas kvaternionais paremtas kampinės pozicijos skaičiavimas, bei nuosekliai jungti PID valdikliai valdymui.
Valdymo komandos siunčiamos per mobiliuosius 3G, 4G tinklus.
Šiuo metu atliekami pirmieji skrydžio bandymai.\\
\rule{0mm}{5mm}\textbf{Technologijos:} ARM-Cortex M4, GCC, git, MEMS, Raspberry pi, GSM.}


\section{Programavimo įgūdžiai}

\skillgroup{Programavimo kalbos}
{
	\textit{Assembler, C, C++} - 4 metų patirtis profesinėje srityje, 8 įskaitant ir hobby veiklą.\\
	\textit{Octave, Matlab, Python, Bash} - Dažnas pagalbinių ar pagrindinių scriptų rašymas profesinėje ir asmeninėje srityje.\\
	\textit{Java, Ruby, html, css, sql, latex} - Naudota asmeniniais tikslais \\

}

\skillgroup{Kiti įgūdžiai}
{
	\textit{Linux} - 7 metai asmeninio naudojimo, taip pat linux tvarkyklių rašymas profesiniais ir asmeniniais tikslais.\\
	\textit{git, svn} - 4 metus naudota profesiniais bei asmeniniais tikslais.\\
	\textit{vim} - 2 metus asmeniniais ir profesiniais tikslais\\
}


\section{Interesai}

\interestsgroup{
	\interest{Įterptinės sistemos, 3D grafika, mašinų mokymasis;}
	\interest{"Encounter" urbanistiniai žaidimai;}
	\interest{Vairavimas;}
}














\end{document}



