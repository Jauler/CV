%************************************
%
% Cirriculum Vitae - Rytis Karpuška
%
%************************************

%Packages stuff
\documentclass[12]{article}
%%%%%%%%%%%%%%%%%%%%%%%%%%%%%%%%%%%%%%%%%
% Wilson Resume/CV
% Structure Specification File
% Version 1.0 (22/1/2015)
%
% This file has been downloaded from:
% http://www.LaTeXTemplates.com
%
% License:
% CC BY-NC-SA 3.0 (http://creativecommons.org/licenses/by-nc-sa/3.0/)
%
%%%%%%%%%%%%%%%%%%%%%%%%%%%%%%%%%%%%%%%%%

%----------------------------------------------------------------------------------------
%	PACKAGES AND OTHER DOCUMENT CONFIGURATIONS
%----------------------------------------------------------------------------------------

\usepackage[a4paper, hmargin=25mm, vmargin=30mm, top=20mm]{geometry} % Use A4 paper and set margins

\usepackage{fancyhdr} % Customize the header and footer

\usepackage{lastpage} % Required for calculating the number of pages in the document

\usepackage{hyperref} % Colors for links, text and headings

\setcounter{secnumdepth}{0} % Suppress section numbering

%\usepackage[proportional,scaled=1.064]{erewhon} % Use the Erewhon font
%\usepackage[erewhon,vvarbb,bigdelims]{newtxmath} % Use the Erewhon font
\usepackage[utf8]{inputenc} % Required for inputting international characters
\usepackage[T1]{fontenc} % Output font encoding for international characters

\usepackage{fontspec} % Required for specification of custom fonts
\setmainfont[Path = ./fonts/,
Extension = .otf,
BoldFont = Erewhon-Bold,
ItalicFont = Erewhon-Italic,
BoldItalicFont = Erewhon-BoldItalic,
SmallCapsFeatures = {Letters = SmallCaps}
]{Erewhon-Regular}

\usepackage{color} % Required for custom colors
\definecolor{slateblue}{rgb}{0.17,0.22,0.34}

\usepackage{sectsty} % Allows customization of titles
\sectionfont{\color{slateblue}} % Color section titles

\fancypagestyle{plain}{\fancyhf{}\cfoot{\thepage\ of \pageref{LastPage}}} % Define a custom page style
\pagestyle{plain} % Use the custom page style through the document
\renewcommand{\headrulewidth}{0pt} % Disable the default header rule
\renewcommand{\footrulewidth}{0pt} % Disable the default footer rule

\setlength\parindent{0pt} % Stop paragraph indentation

% Non-indenting itemize
\newenvironment{itemize-noindent}
{\setlength{\leftmargini}{0em}\begin{itemize}}
{\end{itemize}}

% Text width for tabbing environments
\newlength{\smallertextwidth}
\setlength{\smallertextwidth}{\textwidth}
\addtolength{\smallertextwidth}{-2cm}

\newcommand{\sqbullet}{~\vrule height 1ex width .8ex depth -.2ex} % Custom square bullet point definition

%----------------------------------------------------------------------------------------
%	MAIN HEADER COMMAND
%----------------------------------------------------------------------------------------

\renewcommand{\title}[1]{
{\huge{\color{slateblue}\textbf{#1}}}\\ % Header section name and color
\rule{\textwidth}{0.5mm}\\ % Rule under the header
}

%----------------------------------------------------------------------------------------
%	JOB COMMAND
%----------------------------------------------------------------------------------------

\newcommand{\job}[6]{
\begin{tabbing}
\hspace{2cm} \= \kill
\textbf{#1} \> \href{#4}{#3} \\
\textbf{#2} \>\+ \textit{#5} \\
%\begin{minipage}{\smallertextwidth}
%\vspace{2mm}
#6
%\end{minipage}
\end{tabbing}
%\vspace{2mm}
}


%----------------------------------------------------------------------------------------
%	HOBBY PROJECT COMMAND
%----------------------------------------------------------------------------------------

\newcommand{\hobbyproject}[2]{
\begin{tabbing}
\hspace{2cm} \= \kill
\textbf{#1} \> #2 \\
\end{tabbing}
}


%----------------------------------------------------------------------------------------
%	SKILL GROUP COMMAND
%----------------------------------------------------------------------------------------

\newcommand{\skillgroup}[2]{
\begin{tabbing}
\hspace{5mm} \= \kill
\sqbullet \>\+ \textbf{#1} \\
\begin{minipage}{\smallertextwidth}
\vspace{2mm}
#2
\end{minipage}
\end{tabbing}
}

%----------------------------------------------------------------------------------------
%	INTERESTS GROUP COMMAND
%-----------------------------------------------------------------------------------------

\newcommand{\interestsgroup}[1]{
\begin{tabbing}
\hspace{5mm} \= \kill
#1
\end{tabbing}
\vspace{-10mm}
}

\newcommand{\interest}[1]{\sqbullet \> \textbf{#1}\\[3pt]} % Define a custom command for individual interests

%----------------------------------------------------------------------------------------
%	TABBED BLOCK COMMAND
%----------------------------------------------------------------------------------------

\newcommand{\tabbedblock}[1]{
\begin{tabbing}
\hspace{2cm} \= \hspace{4cm} \= \kill
#1
\end{tabbing}
}



\begin{document}

\title{Rytis Karpuška}


\parbox{0.5\textwidth}{ % First block
\begin{tabbing} % Enables tabbing
\hspace{3cm} \= \hspace{4cm} \= \kill % Spacing within the block
{\bf Address:} \> Paukščių tako 72,\\ % Address line 1
\> Vilnius, LT12189\\ % Address line 2
{\bf Date of birth:} \> 1992-09-30\\ % Date of birth 
\end{tabbing}}
\hfill % Horizontal space between the two blocks
\parbox{0.5\textwidth}{ % Second block
\begin{tabbing} % Enables tabbing
\hspace{3cm} \= \hspace{4cm} \= \kill % Spacing within the block
{\bf Phone.:} \> +37061476947 \\ % Mobile phone
{\bf E-mail:} \> \href{mailto:rytis.karpuska@gmail.com}{rytis.karpuska@gmail.com} \\ % Email address
\end{tabbing}}

\section{About me}

Since my childhood I was interested in exact sciences.
I got interested in physics when I was 11-12 years old, however, over the time, electronics became the most interesting branch of physics to me.
I started soldering simple devices like LED blinkers, audio amplifiers etc.
After a few years I came across microcontrollers and then I wrote my first assembler program which was capable of turning on LED.
That got me hooked up and so, since then, I have been working on projects related to programming in free time contantly.

\section{Education}

\tabbedblock{
	\bf{2011 - 2015} \> University of Vilnius Bachelors degree in Software Engineering - \href{http://www.vu.lt}{University of Vilnius} \\[5pt]
	\>First and second semesters - 90+\% Average\\
	\>\+
}

\tabbedblock{
	\bf{2011} \> Andrew Ng online course "Machine Learning". \textit{http://ml-class.org}\\
	\>\+
}

\section{Professional experience}

\job
{2011 -}{2015}
{JSC "Elektromotus", Žirmūnų g. 68, Vilnius, Lietuva}
{http://www.elektromotus.eu}
{Architect, Software developer}
{
}

\job
{2015 -}{now}
{JSC "Neurotechnology", Laisvės pr. 125A, Vilnius, Lietuva}
{http://www.neurotechnology.com}
{Software developer}
{
}

\section{Projects}

\job
{2007}{}
{Obstacle avoiding robot for "Infobalt 2007" excibition.}
{http://blog.elektronika.lt/robotai/2007/10/28/reportazai-is-infobalt2007/}
{http://blog.elektronika.lt/robotai/2007/10/28/reportazai-is-infobalt2007/}
{This is one of the most interesting works from my early ages.
Robot was controlled by AVR ATmega microcontroller and had some IR sensors for obstacle detection. \\
\rule{0mm}{5mm}\textbf{Technologies:} AVR, GCC.}


\job
{2011 -}{2012}
{Battery Management System "Emus BMS"}
{http://www.elektromotus.lt/lt/produktai/bms.html}
{http://www.elektromotus.lt/lt/produktai/bms.html}
{
This Battery Management System is JSC "Elektromotus" product.
I was one of the main programmers of this system in 2011.
One of the most notable uses of this system is "ACCIONA" team in Dakar rally.\\
\rule{0mm}{5mm}\textbf{Technologies:} AVR, svn, GCC, Linux, Qt4.}

\job
{2012 - }{2013}
{Electric "Smart fourtwo" conversion kit ECU.}
{http://grynas.delfi.lt/tv/lietuviu-perdarytas-elektromobilis-100-km-nuvaziuoja-uz-7-litus.d?id=61776195}
{http://grynas.delfi.lt/tv/lietuviu-perdarytas-elektromobilis-100-km-nuvaziuoja-uz-7-litus.d?id=61776195}
{
During this JSC "Elektromotus" project electric conversion kit for "Smart fortwo" car has been developed.
I have designed and programmed firmware for main ECU unit which controlled engine, gearbox, cooling and communicated to other systems over CAN bus.\\
\rule{0mm}{5mm}\textbf{Technologies:} ARM-Cortex M3, Linux, GCC, git, CAN.}

\job
{2013 -}{2015}
{Quadcopter controlled over internet.}
{https://github.com/jauler/Quadcopter}
{https://github.com/jauler/Quadcopter}
{
This is my current personal freetime project.
I have developed all of the flight and stability algorithms.
Currently first flight tests are being performed.\\
\rule{0mm}{5mm}\textbf{Technologies:} ARM-Cortex M4, GCC, git, MEMS, Raspberry pi, GSM.}

\job
{2014 -}{2015}
{Sensor network for Norvegian railways contact wire monitoring.}
{}
{}
{
During this JSC "Elektromotus" project a custom wireless sensor network system has been developed for acceleration and rotation measurements of contact wire.
Due to high requirements for power consumptions - custom variable latency radio network protocol has been developed to allow radio hardware to be in low-power mode most of the time.
Modified version of sliding window protocol has been impleneted to allow reliable data transmission over the network.
Communication stack has been implemented in ARM Cortex-M3 and also as a linux kernel device driver.
The whole system has the ability to be controlled over the internet.\\
\rule{0mm}{5mm}\textbf{Technologies:} ARM-Cortex M3, Linux kernel device drivers, GCC, git, MEMS, 2.4Ghz radio, Raspberry pi, GSM, VPN, VPS.}

\job
{2016 -}{now}
{MegaMatcher Accelerator Cluster}
{https://www.neurotechnology.com/megamatcher-accelerator.html}
{}
{
During this JSC "Neurotechnology" project, large scale distributed biometric identification system was developed based on principles of Virtual Synchrony model and Neurotechnology's proprietary biometric algorithms.
Linear scalability has been achieved in capacity and speed with cluster sizes up to 30 nodes.
The cluster is capable of storing 160 million records (10 fingerprints per record) while matching at the speed of 300 million records per second.
The system is a CP system which achieves sequential consistency on per subject basis at the cost of being unavailable if quorum is lost.
Separate asynchronous replication protocol was implemented to provide weaker consistency guarantees, but achieve high availability.\\
\rule{0mm}{5mm}\textbf{Technologies:} Linux, Docker, Kubernetes, C++, mysql, corosync.}


\section{Skills}

\skillgroup{Programming languages}
{
	\textit{Assembler, C, C++} - 4 years experience in professional environment, 8 years including hobby usage.\\
	\textit{Octave, Matlab, Python, Bash} - Frequent usage for professional and hobby needs.\\
	\textit{Java, Ruby, html, css, sql, latex} - Used for personal needs. \\

}

\skillgroup{Other}
{
	\textit{Linux} - 7 years usage for personal purposes, some experience in linux device drivers development.\\
	\textit{git, svn} - 4 years profesional and personal usage.\\
	\textit{vim} - 2 years profesional and personal usage.\\
}


\section{Interests}

\interestsgroup{
	\interest{Embedded systems, 3D graphics, machine learning;}
	\interest{"Encounter" urban games;}
}

\end{document}



