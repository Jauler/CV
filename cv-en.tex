%************************************
%
% Cirriculum Vitae - Rytis Karpuška
%
%************************************

%Packages stuff
\documentclass[12]{article}
%%%%%%%%%%%%%%%%%%%%%%%%%%%%%%%%%%%%%%%%%
% Wilson Resume/CV
% Structure Specification File
% Version 1.0 (22/1/2015)
%
% This file has been downloaded from:
% http://www.LaTeXTemplates.com
%
% License:
% CC BY-NC-SA 3.0 (http://creativecommons.org/licenses/by-nc-sa/3.0/)
%
%%%%%%%%%%%%%%%%%%%%%%%%%%%%%%%%%%%%%%%%%

%----------------------------------------------------------------------------------------
%	PACKAGES AND OTHER DOCUMENT CONFIGURATIONS
%----------------------------------------------------------------------------------------

\usepackage[a4paper, hmargin=25mm, vmargin=30mm, top=20mm]{geometry} % Use A4 paper and set margins

\usepackage{fancyhdr} % Customize the header and footer

\usepackage{lastpage} % Required for calculating the number of pages in the document

\usepackage{hyperref} % Colors for links, text and headings

\setcounter{secnumdepth}{0} % Suppress section numbering

%\usepackage[proportional,scaled=1.064]{erewhon} % Use the Erewhon font
%\usepackage[erewhon,vvarbb,bigdelims]{newtxmath} % Use the Erewhon font
\usepackage[utf8]{inputenc} % Required for inputting international characters
\usepackage[T1]{fontenc} % Output font encoding for international characters

\usepackage{fontspec} % Required for specification of custom fonts
\setmainfont[Path = ./fonts/,
Extension = .otf,
BoldFont = Erewhon-Bold,
ItalicFont = Erewhon-Italic,
BoldItalicFont = Erewhon-BoldItalic,
SmallCapsFeatures = {Letters = SmallCaps}
]{Erewhon-Regular}

\usepackage{color} % Required for custom colors
\definecolor{slateblue}{rgb}{0.17,0.22,0.34}

\usepackage{sectsty} % Allows customization of titles
\sectionfont{\color{slateblue}} % Color section titles

\fancypagestyle{plain}{\fancyhf{}\cfoot{\thepage\ of \pageref{LastPage}}} % Define a custom page style
\pagestyle{plain} % Use the custom page style through the document
\renewcommand{\headrulewidth}{0pt} % Disable the default header rule
\renewcommand{\footrulewidth}{0pt} % Disable the default footer rule

\setlength\parindent{0pt} % Stop paragraph indentation

% Non-indenting itemize
\newenvironment{itemize-noindent}
{\setlength{\leftmargini}{0em}\begin{itemize}}
{\end{itemize}}

% Text width for tabbing environments
\newlength{\smallertextwidth}
\setlength{\smallertextwidth}{\textwidth}
\addtolength{\smallertextwidth}{-2cm}

\newcommand{\sqbullet}{~\vrule height 1ex width .8ex depth -.2ex} % Custom square bullet point definition

%----------------------------------------------------------------------------------------
%	MAIN HEADER COMMAND
%----------------------------------------------------------------------------------------

\renewcommand{\title}[1]{
{\huge{\color{slateblue}\textbf{#1}}}\\ % Header section name and color
\rule{\textwidth}{0.5mm}\\ % Rule under the header
}

%----------------------------------------------------------------------------------------
%	JOB COMMAND
%----------------------------------------------------------------------------------------

\newcommand{\job}[6]{
\begin{tabbing}
\hspace{2cm} \= \kill
\textbf{#1} \> \href{#4}{#3} \\
\textbf{#2} \>\+ \textit{#5} \\
%\begin{minipage}{\smallertextwidth}
%\vspace{2mm}
#6
%\end{minipage}
\end{tabbing}
%\vspace{2mm}
}


%----------------------------------------------------------------------------------------
%	HOBBY PROJECT COMMAND
%----------------------------------------------------------------------------------------

\newcommand{\hobbyproject}[2]{
\begin{tabbing}
\hspace{2cm} \= \kill
\textbf{#1} \> #2 \\
\end{tabbing}
}


%----------------------------------------------------------------------------------------
%	SKILL GROUP COMMAND
%----------------------------------------------------------------------------------------

\newcommand{\skillgroup}[2]{
\begin{tabbing}
\hspace{5mm} \= \kill
\sqbullet \>\+ \textbf{#1} \\
\begin{minipage}{\smallertextwidth}
\vspace{2mm}
#2
\end{minipage}
\end{tabbing}
}

%----------------------------------------------------------------------------------------
%	INTERESTS GROUP COMMAND
%-----------------------------------------------------------------------------------------

\newcommand{\interestsgroup}[1]{
\begin{tabbing}
\hspace{5mm} \= \kill
#1
\end{tabbing}
\vspace{-10mm}
}

\newcommand{\interest}[1]{\sqbullet \> \textbf{#1}\\[3pt]} % Define a custom command for individual interests

%----------------------------------------------------------------------------------------
%	TABBED BLOCK COMMAND
%----------------------------------------------------------------------------------------

\newcommand{\tabbedblock}[1]{
\begin{tabbing}
\hspace{2cm} \= \hspace{4cm} \= \kill
#1
\end{tabbing}
}


\usepackage{multicol}
\usepackage{enumitem}
\usepackage{xurl}


\begin{document}

\title{Rytis Karpuška}


\parbox{0.5\textwidth}{ % First block
\begin{tabbing} % Enables tabbing
\hspace{3cm} \= \hspace{4cm} \= \kill % Spacing within the block
{\bf Location:} \> Vilnius, Lithuania\\ % Address line 1
{\bf Date of birth:} \> 1992-09-30\\ % Date of birth
\end{tabbing}}
\hfill % Horizontal space between the two blocks
\parbox{0.5\textwidth}{ % Second block
\begin{tabbing} % Enables tabbing
\hspace{3cm} \= \hspace{4cm} \= \kill % Spacing within the block
{\bf Phone.:} \> +37061476947 \\ % Mobile phone
{\bf E-mail:} \> \href{mailto:rytis.karpuska@gmail.com}{rytis.karpuska@gmail.com} \\ % Email address
\end{tabbing}}

\section{About me}

Since my childhood I was interested in exact sciences like math, physics, etc.
When I was 11-12 years old, physics attracted me the most, however, over time, electronics became the most interesting branch to me.
I started soldering simple devices like LED blinkers, audio amplifiers etc.
Few years in, I came across microcontrollers and after writing my first assembler "program" which was berely capable of turning on LED - I got hooked on programming.
This was turning point in my life.
Since then, I have been working on projects related to programming in professional and hobby settings constantly.

\section{Education}

\tabbedblock{
	\bf{2011 - 2015} \> University of Vilnius Bachelors degree in Software Engineering - \href{http://www.vu.lt}{University of Vilnius} \\[5pt]
	\>First and second semesters - 90+\% Average
	\>\+
}

\section{Experience}

\hobbyproject
{2007}{Hobby project}
\begin{itemize}[leftmargin=2cm,topsep=-0.5cm]

\item \textbf{Obstacle avoiding robot for ”Infobalt 2007” excibition}

This is one of the most interesting works from my early ages. Robot was controlled by AVR AT-
mega microcontroller and had some IR sensors for obstacle detection.

\rule{0mm}{5mm}\textbf{Technologies:} AVR, GCC, Servo.

\rule{0mm}{5mm}\textbf{Reference:} \url{http://blog.elektronika.lt/robotai/2007/10/28/reportazai-is-infobalt2007/}

\end{itemize}




\job
{2011 -}{2015}
{JSC "Elektromotus", Žirmūnų g. 68, Vilnius, Lietuva}
{http://www.elektromotus.eu}
{Software engineer}
{}
\begin{itemize}[leftmargin=2cm,topsep=-0.5cm]

\item \textbf{Battery Management System "Emus BMS"}

This Battery Management System is JSC "Elektromotus" product.
I was one of the main developers of this system in 2011.

\rule{0mm}{5mm}\textbf{Technologies:} AVR, svn, GCC, Linux, Qt4.

\rule{0mm}{5mm}\textbf{Reference:} \url{https://emusbms.com/product-category/g1-control-unit}

\item \textbf{Electric "Smart fourtwo" conversion kit ECU.}

During this JSC "Elektromotus" project electric conversion kit for "Smart fortwo" car has been developed.
I have designed and programmed firmware for main ECU unit which controlled engine, gearbox, cooling and communicated to other systems over CAN bus.

\rule{0mm}{5mm}\textbf{Technologies:} ARM-Cortex M3, Linux, GCC, git, CAN.

\rule{0mm}{5mm}\textbf{Reference:} \url{http://grynas.delfi.lt/tv/lietuviu-perdarytas-elektromobilis-100-km-nuvaziuoja-uz-7-litus.d?id=61776195}

\item \textbf{Sensor network for Norwegian railways contact wire monitoring.}

During this JSC "Elektromotus" project I and one of my colleagues developed a custom wireless sensor network system for acceleration and rotation measurements of contact wire.
Due to high requirements for power consumption - we developed custom variable latency radio network protocol to allow radio hardware to be in low-power mode most of the time.
The system was capable of measuring and reporting data for over a week on single charge even if sensors are scattered over 1 km range.

\rule{0mm}{5mm}\textbf{Technologies:} ARM-Cortex M3, Linux kernel device drivers, GCC, git, MEMS, 2.4Ghz radio, Raspberry pi, GSM, VPN, VPS.

\rule{0mm}{5mm}\textbf{Reference:} \url{https://www.sciencedirect.com/science/article/pii/S2214399816300121}

\end{itemize}

\job
{2015 -}{now}
{JSC "Neurotechnology", Laisvės pr. 125A, Vilnius, Lietuva}
{http://www.neurotechnology.com}
{Software engineer}
{}

\begin{itemize}[leftmargin=2cm,topsep=-0.5cm]
\item \textbf{MegaMatcher Accelerator Cluster}

During this JSC "Neurotechnology" project, I, together with one of my  colleagues, developed large scale distributed biometric identification system on principles of virtual synchrony model and Neurotechnology's proprietary biometric algorithms.
We achieved linear scalability in capacity and speed with cluster sizes up to 30 nodes.
The cluster is capable of storing 1.6 billion fingerprint records while matching at the speed of 3 billion fingerprints per second when used without GPU acceleration.
Other modalities, like Face and Iris are also supported.
The system has been successfully deployed for multiple countries national ID and border control projects.

\rule{0mm}{5mm}\textbf{Technologies:} Linux, Docker, Kubernetes, helm, C++, mysql, corosync, Neurotechnology Biometric SDK.

\rule{0mm}{5mm}\textbf{Reference:} \url{https://www.neurotechnology.com/megamatcher-accelerator.html}

\end{itemize}

\section{Skills}

\begin{multicols}{2}

\strong{Tools and Technologies}
\begin{itemize}
	\item{Docker, LXC}
	\item{Kubernetes, Helm}
	\item{Apache kafka, Apache zookeeper}
	\item{MySQL, Galera}
	\item{Jenkins}
	\item{git, svn}
	\item{GCC, GDB, GNU make, cmake}
	\item{Linux, Debian}
\end{itemize}

\columnbreak

\strong{Programming languages}
\begin{itemize}
	\item{C, C++}
	\item{Assembler}
	\item{Python}
	\item{Bash}
\end{itemize}


\strong{Embedded Systems}
\begin{itemize}
	\item{STM32F series microcontrollers}
	\item{Atmel AVR series microcontrollers}
	\item{Cross-compilation with GCC, GDB}
\end{itemize}


\vfill
\end{multicols}


\clearpage

\section{More notable contributions to open source projects}

\skillgroup {OpenAPI}
{
Multiple contributions to OpenAPI C++ code generator:
\begin{itemize}
	\item \url{https://github.com/OpenAPITools/openapi-generator/pull/732}
	\item \url{https://github.com/OpenAPITools/openapi-generator/pull/731}
	\item \url{https://github.com/OpenAPITools/openapi-generator/pull/640}
	\item \url{https://github.com/OpenAPITools/openapi-generator/pull/631}
\end{itemize}
}


\skillgroup {Corosync}
{
Multiple contributions to Corosync Cluster Engine:
\begin{itemize}
	\item \url{https://github.com/corosync/corosync/pull/335}
	\item \url{https://github.com/corosync/corosync/pull/321}
	\item \url{https://github.com/corosync/corosync/pull/320}
	\item \url{https://github.com/corosync/corosync/pull/300}
	\item \url{https://github.com/corosync/corosync/pull/297}
\end{itemize}
}

\skillgroup {Zookeeper C++ client library}
{
Multiple contributions to zookeeper-cpp library
\begin{itemize}
	\item \url{https://github.com/tgockel/zookeeper-cpp/pull/111}
	\item \url{https://github.com/tgockel/zookeeper-cpp/pull/110}
	\item \url{https://github.com/tgockel/zookeeper-cpp/pull/109}
\end{itemize}
}

\skillgroup{Picom}
{
Implementation of a new feature for X11 compositor which helps to lower eye strain
\begin{itemize}
	\item \url{https://github.com/yshui/picom/pull/247}
\end{itemize}
}



\section{Interests}

\interestsgroup{
	\interest{Large scale distributed computing, Open Source, Embedded systems, Linux}
	\interest{White hat hacking and CTF competitions: \url{https://ctftime.org/user/75218}}
	\interest{"Encounter" urban games: \url{http://vilnius.en.cx/UserDetails.aspx?uid=1405002}}
}

\end{document}



